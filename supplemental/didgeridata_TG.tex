\documentclass[letterpaper,12pt]{scrartcl}
\usepackage{stem}
\usepackage{multirow}

%adjustments for table cell padding
%change these if your table elements don't fit right
\setlength{\tabcolsep}{10pt}
\renewcommand{\arraystretch}{1.5}

\makeatletter
\title{
%EDIT YOUR TITLE HERE
%====================
Didgeridata
%====================
}\let\myTitle\@title

%Define 3D Model
%\newcommand{\KitName}{
%===================================
%Didgeridata
%===================================
%}

%Define Sensor Type (if needed)
%\newcommand{\SensorName}{
%===================================
%
%===================================
%}

\makeatletter
\let\runtitle\@title
\makeatother
\rhead{\runtitle: 
%EDIT THE TYPE OF DOCUMENT HERE (Options: see below; comment out what you do not use)
%Do not modify the options
%============================================================================================
Teacher Guide
%Student Handout
%Student Assessment
%============================================================================================
}

\begin{document}
\maketitle
\thispagestyle{fancy}
%=================================
%THIS IS THE START OF THE DOCUMENT
%=================================

\section*{Introduction} %asterisk hides section numbering
In this lesson, students will investigate climate data trends and then use their findings as inspiration for a musical composition. Each musical composition will be played on student-constructed wind-powered instruments - the didgeridoo and the paixiao. The didgeridoo is an ancient instrument created by aboriginal Australians while the paixiao is a form of pan flute whose origins are located near Hong Kong. \\

\section*{Guiding Questions}
%boilerplate language; do not edit.
The following questions may help to guide students' thoughts as they progress through this lesson and associated activities. 
\begin{itemize}
	\item What trends, if any, did you expect to see in the climate data? 
	\item Did you find any climate data trends that were unexpected?
	\item How might we use music to build awareness of climate issues?
\end{itemize}

\section*{Learning Objectives}
\begin{itemize}
	\item Students will investigate and interpret climate data.
	\item Students will construct PVC musical instruments using basic construction techniques. 
	\item Students will perform a data-inspired musical composition using their instruments. 
\end{itemize}


\section*{Timeline}
This lesson will require approximately two to three hours spread over three phases. Phase One will include an exploration and interpretation of climate data. Phase Two will include construction, testing, and refinement of PVC instruments. Phase Three will include composition and performance of a musical piece to demonstrate some aspect of the data in a creative way. 

%More optional styles:
%This lesson will require approximately two 50-minute class periods.
%This lesson will require approximately thirty minutes.
%This lesson will require approximately one hour.   

\section*{Teacher Notes}
Students should work in teams of two to construct one didgeridoo and one paixiao per team. For example, in a class of twenty students, there will be ten pairs creating ten didgeridoos total and ten paixiaos total. Each student may leave with one instrument to continue practice and play at home. 

\section*{Procedure}
\begin{enumerate}
	\item Introduce the lesson and divide students into teams of two. 
	\item Phase One: Introduce students to a few selected climate data sources to get them started with their exploration of climate data. Students should be looking for trends or patterns and should be encouraged to take notes on their discoveries which would include any findings they think are interesting, relevant, or useful. Students may, depending on technology availability, be encouraged to explore as a group using handouts, a single classroom computer, or team computers. Students should be encouraged to share their findings among peers to stimulate discussion.
	\item After the students have conducted their investigation, carry out a discussion and ensure every student has some ideas about what they found interesting. Select some students to share aloud with the class what they found interesting. Perhaps use this as an opportunity to explore what students know about climate change.
	\item Phase Two: Introduce the didgeridoo and the paixiao. Show videos, play example instruments, bring in a talented musician as a guest, or use other methods to raise student familiarity with these instruments. Discuss history, significance, anatomy, use of wind to play the instrument, or other interesting points to engage students. 
	\item Tell the students they will be using basic materials to construct their own instruments. Make the materials available for student use and guide them to safely construct one didgeridoo and one paixiao per team. Check frequently with each team to ensure safe practices are followed. Student should wear appropriate safety gear during this phase. Students should create comfortable-to-play instruments and should practice getting the basic correct sounds from each instrument. Alcohol pads may be used as needed to sterilize instrument mouthpieces before use.  
	\item Phase Three: Students should reflect on their earlier data investigation, their constructed PVC instruments, and how they might use the instruments to create a composition that reflects some aspect of an interesting finding from the data. Students should, in teams, play with some musical ideas to come up with a composition. They may be guided to create a custom notation system to record their idea on paper. This phase concludes with a successful performance by the student teams. Performances should be introduced with a brief explanation of the selected data, the interesting finding(s), and an explanation of how their performance was inspired by the data.
	\item Students may take their instruments home for additional play and practice! 
\end{enumerate}

\section*{Skills \& Practices}
This lesson addresses the following NGSS Science and Engineering skills and practices:
%Use all that apply
\begin{itemize}
	\item Asking questions
	%\item Defining problems
	\item Developing and using models
	%\item Planning and carrying out investigations
	\item Analyzing and interpreting data
	%\item Using mathematics, information and computer technology, and computational thinking
	\item Constructing explanations 
	%\item Designing solutions
	%\item Engaging in argument from evidence
	\item Obtaining, evaluating, and communicating information
\end{itemize}

\section*{Extension Opportunities} %Brief statement
Students may conduct an online investigation of other climate data, create novel musical instruments, and create additional performances. Students should explore their instrument construction methods to investigate frequency limits.


%=================================
%THIS IS THE END OF THE DOCUMENT
%=================================
\end{document}