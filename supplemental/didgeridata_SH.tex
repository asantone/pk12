\documentclass[letterpaper,12pt]{scrartcl}
\usepackage{stem}

%adjustments for table cell padding
%change these if your table elements don't fit right
\setlength{\tabcolsep}{10pt}
\renewcommand{\arraystretch}{1.5}

\makeatletter
\title{
%EDIT YOUR TITLE HERE
%====================
Didgeridata
%====================
}\let\myTitle\@title

\makeatletter
\let\runtitle\@title
\makeatother
\rhead{\runtitle: 
%EDIT THE TYPE OF DOCUMENT HERE (Options: see below; comment out what you do not use)
%Do not modify the options
%============================================================================================
%Teacher Guide
Student Handout
%Student Assessment
%============================================================================================
}

\begin{document}
%\maketitle
\thispagestyle{fancy}
%=================================
%THIS IS THE START OF THE DOCUMENT
%=================================
\noindent Name \& Date:\\ %Name/Date header 
\rule{\textwidth}{1pt}    %horizontal line 
 
\section*{Procedure}
\begin{enumerate}
	\item Form groups of two. 
	\item Following guidance from your teacher, explore climate data and make notes about any interesting findings, trends, unexpected surprises, or notable features of the data set or visualization. Discuss with your team what you think is interesting and what you might want to make sure that other people understand.
	\item Listen carefully about two interesting musical instruments: the didgeridoo and the paixiao. What is the same between these instruments? What is different? 
	\item Use hand tools and basic construction methods to create one didgeridoo and one paixiao per team. Test as you go and remember to measure twice and cut once! Your teacher will provide basic guidance on how to build these instruments. This is a great time to be creative with decoration as well! Make sure you play each instrument to ensure it works!
	\item After you make the instruments, start thinking about how you might represent some of your interesting findings about the data through music. Can you communicate what you learned through sound? Talk with your teammate to put together a plan for a performance and make sure you can write down what you'll do. You may need to create a new way to record these instructions on paper but make sure you and your teammate can read the notes!
	\item Lastly, show everyone what you know! Explain what you found interesting in the data and then play a song for the group. You'll need to explain how your song is related to the data.
	
	\clearpage
	\section{Questions}
	\begin{enumerate}
		\item What were your most interesting observations from the climate data? \vspace{50mm}
		
		\item Explain how the didgeridoo sounds to you. Do you like it? Explain. \vspace{20mm}
		\item Explain how the paixiao sounds to you. Do you like it? Explain. \vspace{20mm}
		\item What characteristics of the instruments did you use to your advantage in creating a musical performance?  \vspace{20mm}
		\item What was the most difficult part of this lesson for you? Explain. \vspace{20mm}
		\item What was your favorite part of this lesson? Explain. 
		
	\end{enumerate}
	
	
\end{enumerate}


%=================================
%THIS IS THE END OF THE DOCUMENT
%=================================
\end{document}